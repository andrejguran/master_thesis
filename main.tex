\documentclass{article}

\usepackage{amsmath}
\usepackage{amsfonts}
\usepackage[utf8]{inputenc}

\setlength{\parskip}{1em}

\begin{document}
\section{Generative Adversarial Networks}


Introduced in 2014 \cite{goodfellow_generative_2014}, generative adversarial  networks, so-called GANs, have been the focus of countless research papers and creative implementations.

We can train them to generate new samples from a given distribution without the difficulties of approximating likelihood functions. In order to do so, we train two neural networks to compete with each other. The discriminator network is trained to tell apart real samples from "fake" samples, while the generator network tries to generate samples that will fool the discriminator.

\subsection{Training Objective}
Given input data with distribution $p(x)$, the generator $G$ is a neural network, that maps random input noise $z$ to the input space, as $G(z, \theta_{g})$, learning the model distribution $p\hat{}(x)$. The discriminator $D$ is a second neural network $D(x, \theta_{d})$ with a single scalar output, classifying the input as real, sampled from $p(x)$, or as fake, sampled from the model distribution $p\hat{}(x)$. 

The training objective of the discriminator is to maximize the probability of assigning the correct label, while the objective of the generator is to minimize this probability. The loss function of the GAN can be described as the minimax objective,

\begin{equation}
\underset{G}{\mathrm{min}} \ \underset{D}{\mathrm{max}} \ V(D,G) = \mathbb{E}_{x \sim p_{data}(x)}[\log D(x)] + \mathbb{E}_{z \sim p_{z}(z)}[\log 1 - D(G(z))]
\label{eq:minimax}
\end{equation}.

In order to prevent vanish gradients, as a result of the discriminator saturating by confidently classifying the samples before the generator's update, \cite{goodfellow_generative_2014} suggests $G$ to be trained to maximize $\mathbb{E}_{z \sim p_{z}(z)}[\log D(G(z))]$ instead.

\subsection{Training Difficulties}
In theory, the minimax game described in equation \ref{eq:minimax} is played until generator has perfectly modeled the distribution $p(x)$, so that discriminator classifies the authenticity of the samples at random. 

In reality, finding the right balance between the two networks is one of the difficulties when training a GAN. If discriminator gets too good at determining which samples are fake, then generator has no chance to learn the distribution. On the other hand, if generator is updated too much, it can collapse too many values of $z$ to the same value of $x$ to have enough diversity to model the distribution.

\section{GAN Comparison}
Lot of papers and open-source repositories have been published since the appearance of the original GAN paper. I have reviewed and tested several of them, in order to find architectures and code applicable to the problem of generating and modifying fashion images. 

\begin{table}
\centering
\begin{tabular}{l*{4}{c}}
Network Characteristics & pix2pix &	StarGAN	& MUNIT	& FaderNetworks \\
\hline
Supervised Training		& Yes & No & No & No \\
Multi-Domain  			& No  & Yes & No & No \\
Multi-Modal				& No & No & Yes & No \\
Latent Representations	& No & No & Yes & Yes \\
\end{tabular}
\caption{\label{tab:gan_comp}\textbf{Comparison of existing GAN models based on their characteristics.} Supervised training means that the dataset consists of image pairs, e.g: a dress and a person wearing the dress. Multi-Domain networks are able to train one model to change multiple attributes. Multi-Modal networks are able to generate more than one possible output. Networks with latent representations try to model the training data in a latent space, as opposed to pixel space.}
\end{table}

Based on various characteristics, compared in Table \ref{tab:gan_comp}, I have evaluated  5 networks: pix2pix \cite{isola_image--image_2016}, CycleGAN \cite{zhu_unpaired_2017}, StarGAN \cite{choi_stargan:_2017}, MUNIT \cite{huang_multimodal_2018} and FaderNetworks \cite{lample_fader_2017}. I have compared the following attributes for each of the mentioned networks:
\begin{itemize}
\item Supervised/Unsupervised learning: Describes if the network needs a dataset consisting of paired images, such as the same skirt in a short and log version.
\item Multi-Domain: Describes if the network is able to train only one model for different domain modifications, or if each domain requieres its own trained generator.
\item Multi-Modal: Describes if the network is able to generate different versions from the same input.
\item Latent representation: Describes if the network trains directly in pixel-space or uses latent representation of the data, usually content and style of the images.
\end{itemize}



\bibliographystyle{apalike}
\bibliography{Zotero}
\end{document}
